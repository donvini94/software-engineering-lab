%% LaTeX2e class for seminar theses
%% sections/conclusion.tex
%% 
%% Karlsruhe Institute of Technology
%% Institute for Program Structures and Data Organization
%% Chair for Software Design and Quality (SDQ)
%%
%% Dr.-Ing. Erik Burger
%% burger@kit.edu

\section{Fazit} 
Alle im Rahmen des Projekts gesetzten Ziele wurden erfolgreich erreicht. Am Ende der Pipeline werden die Ergebnisse der Simulation in Form von leicht verständlichen Grafiken bereitgestellt, die eine schnelle Interpretation der Simulationsergebnisse erlauben.

Es gibt jedoch noch Optimierungspotenzial, insbesondere im Umgang mit den bereitgestellten Traces. Die GitLab-CI-Pipeline könnte erweitert werden, um je nach Branch-Namen unterschiedliche Simulationen durchzuführen oder verschiedene Traces automatisch zu verwenden. Weiterhin könnte man bereits das Sammeln der Laufzeitdaten in die Pipeline integrieren, sodas die Simulationen immer den aktuellsten Stand des Projektes benutzen. Diese Erweiterungen könnten die Flexibilität der Pipeline weiter steigern.
